\documentclass[12pt, a4paper]{article}
\usepackage[utf8]{inputenc}
\usepackage{xspace}
\usepackage{amsmath}
\usepackage[francais]{babel}

\input{macros}

\title{\vspace{-5cm}}
\date{}
\begin{document}
\maketitle

\SpecialSection{Introduction}
Étant en seconde année de DUT Informatique, il m'était nécessaire d'effectuer un stage de fin d'études afin de valider le diplôme, mais surtout d'acquérir une première expérience professionnelle dans le domaine de l'informatique et afin de mettre à profit les compétences acquises lors de ces deux années d'étude. Ce stage s'est déroulé du 4 avril 2022 au 24 juin 2022. J'ai été accueilli par l'entreprise \glaz{} dans l'équipe informatique. \newline
\glaz{} est une Fintech, c'est donc une entreprise alliant finance et technologie. En effet \gz{} propose des services financiers tout en utilisant les nouvelles technologies numériques. De ce fait, l'entreprise utilise \slf{} afin de proposer des solutions informatiques de gestion financière pour des analystes financiers ou des gestionnaires de patrimoine.
Les gestionnaires de patrimoine utilisent encore des outils comme des calculatrices ou Excel. Or, la technologie ne cesse d’évoluer, ce qui permet aux outils dédiés aux gestionnaires de patrimoine d'évoluer également.
Ainsi, l'entreprise \glaz{} veut proposer une solution de gestion de relation clientèle pour leur faciliter le travail. C'est pourquoi ma mission était de concevoir et développer, avec l'aide d'un recueil de besoins, un outil de gestion de relation clientèle conçu pour les gestionnaires de patrimoine, dans une équipe de 3 personnes. L'objectif de cet outil étant d'avoir une vision globale du client pour le gestionnaire.
\slf{} étant une solution de gestion de relation client, et étant le noyau de travail de l'entreprise \glaz{}, nous l'avons utilisé pour développer notre outil de gestion et de fidélisation clientèle que l'on appelle "Gespat". Nous allons donc voir comment aider les gestionnaires de patrimoine avec Salesforce.\newline
C'est pourquoi, dans un premier temps, nous allons commencer par présenter l'entreprise \glaz{}, mais également ses différents projets, pour ensuite montrer l'environnement et les outils de travail utilisés par \gz{}, afin de pouvoir rebondir sur le projet auquel j'étais assigné et auquel j'ai contribué.
Dans un second temps, je détaillerais les différentes parties de ma contribution sur le projet Gespat, et notamment les différentes contraintes et les développements effectués pour y répondre.
La fin de ce rapport sera consacrée à une prise de recul sur les différentes compétences acquises pendant ce stage. Je mettrai également en exergue le lien entre le travail effectué et l'apprentissage que j'ai suivi à l'IUT.

\end{document}